\section{Physical Fundamentals of Mechanics}
\subsection{Kinematics}
For a given point undergoing motion, the average and instantaneous velocities are
\[ \vbrack{\vec{v}} = \frac{\Delta \vec x}{\Delta t} \qquad \vec v = \frac{d\vec x}{dt} \]
where $\vec x$ is the displacement vector. Similarly the average and instantaneous accelerations are
\[ \vbrack{\vec a} = \frac{\Delta \vec v}{\Delta t} \qquad \vec a = \frac{d\vec v}{dt} \]
 
The distance covered by a point is given by the integral
\[ s = \int_{t_0}^{t_1} v \, dt \] 
 
The acceleration of a point may be broken down into components tangential and normal to the a trajectory, let $R$ be the radius of curvature at a given point, then the components are:
 \[ a_t = \frac{d v_t}{dt} \qquad a_n = \frac{v^2}{R} \]
 
The angular velocity and accelerations are given by
\[ \vec{\omega} = \frac{d\vec\theta}{dt} \qquad \vec\alpha = \frac{d\vec\omega}{dt} \]
To convert between angular and linear quantities for a rotating body
\[ \vec{v} = (\omega \vec r) \qquad \omega_n = \omega^2 R \qquad |\omega_t| = \alpha R \]
where $R$ is the distance from the rotation action and $\vec r$ is the radius vector.

% Exercise 1
\begin{exercise}

\end{exercise}

\begin{exercise}
\end{exercise}

\begin{exercise}
\end{exercise}

\begin{exercise}
\end{exercise}

\begin{exercise}
\end{exercise}

% Exercise 5
\begin{exercise}
\end{exercise}

\begin{exercise}
\end{exercise}

\begin{exercise}
\end{exercise}

\begin{exercise}
\end{exercise}

\begin{exercise}
\end{exercise}

% Exercise 10
\begin{exercise}
\end{exercise}

\begin{exercise}
\end{exercise}

\begin{exercise}
\end{exercise}

\begin{exercise}
\end{exercise}

\begin{exercise}
\end{exercise}

% Exercise 15
\begin{exercise}
\end{exercise}

\begin{exercise}
\end{exercise}

\begin{exercise}
\end{exercise}

\begin{exercise}
\end{exercise}

\begin{exercise}
\end{exercise}

% Exercise 20
\begin{exercise}
\end{exercise}

\begin{exercise}
\end{exercise}

\begin{exercise}
\end{exercise}

\begin{exercise}
\end{exercise}

\begin{exercise}
\end{exercise}

% Exercise 25
\begin{exercise}
\end{exercise}

\begin{exercise}
\end{exercise}

\begin{exercise}
\end{exercise}

\begin{exercise}
\end{exercise}

\begin{exercise}
\end{exercise}

% Exercise 30
\begin{exercise}
\end{exercise}

\begin{exercise}
\end{exercise}

\begin{exercise}
\end{exercise}

\begin{exercise}
\end{exercise}

\begin{exercise}
\end{exercise}

% Exercise 35
\begin{exercise}
\end{exercise}

\begin{exercise}
\end{exercise}

\begin{exercise}
\end{exercise}

\begin{exercise}
\end{exercise}

\begin{exercise}
\end{exercise}

% Exercise 40
\begin{exercise}
\end{exercise}

\begin{exercise}
\end{exercise}

\begin{exercise}
\end{exercise}

\begin{exercise}
\end{exercise}

\begin{exercise}
\end{exercise}

% Exercise 45
\begin{exercise}
\end{exercise}

\begin{exercise}
\end{exercise}

\begin{exercise}
\end{exercise}

\begin{exercise}
\end{exercise}

\begin{exercise}
\end{exercise}


% Exercise 50
\begin{exercise}
\end{exercise}

\begin{exercise}
\end{exercise}

\begin{exercise}
\end{exercise}

\begin{exercise}
\end{exercise}

\begin{exercise}
\end{exercise}

% Exercise 55
\begin{exercise}
\end{exercise}

\begin{exercise}
\end{exercise}

\begin{exercise}
\end{exercise}