\section{Groups and Representations}
\subsection{Notes}

Before we can discuss Lie algebras, we must discuss groups and their representations.

\begin{definition}[Groups]
	A group is a set $G$ with a binary operation $\cdot$ such that for any $x, y, z \in G$
	\begin{itemize}
		\item $x \cdot y \in G$, that is $G$ is closed under the operation
		\item There exists an identity element $e$ such that $e \cdot x = x \cdot e = x$
		\item There exists an inverse $\inv x \in G$ such that $\inv x \cdot x = x \cdot \inv x = e$
		\item The operation is transitive $(x \cdot y) \cdot z = x \cdot (y \cdot z)$
	\end{itemize}
\end{definition}
	
\begin{example}
	The most important example in physics of a group is the permutation group. Suppose we have a set of three elements $\setb{a,b,c}$. Then the identity element is the permutation that does nothing. More complicated permutations are represented using the following notation
	\[ (12) \thus \text{swap elements 1 and 2} \]
	\[ (123) \thus \text{move element 1 to position 2, element 2 to position 3, etc.} \]
	
	Each closed group of parentheses is called a cycle, because it cycles the positions enclosed. Every permutation can be represented as a set of cycles. The multiplication law is given by concatenation of permutations, that is we apply the rightmost permutation then proceed leftwards, applying permutations as we go. For instance,
	\[ (12)(23) = (123) \]
	
	The permutation group of $n$ elements is denoted $S_n$.
\end{example}

The group of permutations is an example of a transformation group. In general, the set of reversible transformations on an object has a group structure and the groups they form are very important in physics because they include all possible symmetries of a system.

In quantum mechanics, a transformation is given by a unitary operator in Hilbert space. Thus any transformation group can be mapped onto a set of unitary operators, i.e. for every $x \in G$ there exists an operator $D(x)$ which is unitary. For the mapping to make physical sense, it must preserve any multiplication
\[ D(x \cdot y) = D(x) D(y) \]

\begin{definition}[Representations]
	Let $G$ be a group and $V$ a vector space, then a representation of $G$ is a map
	\[ D: G \to GL(V) \]
	such that multiplication is preserved
	\[ D(x \cdot y) = D(x) D(y) \qquad \forall x,y \in G \]
\end{definition}

\begin{example}
	Let $G = \z$ be the integers. Then we may represent the additive group with the map
	\[ D(n) = e^{in\theta} \]
	for any choice of $\theta \in \re$.
\end{example}

\begin{example}
	Consider the permutation group $S_3$ again, we can represent each permutation with a matrix
	\[ \begin{aligned}
		D(e) &= \mqty(\imat{3}) \qquad D(12) &= \mqty(0 & 1 & 0 \\ 1 & 0 & 0 \\ 0 & 0 & 1) \\
		D(23) &= \mqty(1 & 0 & 0 \\ 0 & 0 & 1 \\ 0 & 1 & 0) \qquad D(13) &= \mqty(0 & 0 & 1 \\ 0 & 1 & 0 \\ 1 & 0 & 0) \\
		D(123) &= \mqty(0 & 0 & 1 \\ 1 & 0 & 0 \\ 0 & 1 & 0) \qquad D(132) &= \mqty(0 & 1 & 0 \\ 0 & 0 & 1 \\ 1 & 0 & 0)
	\end{aligned} \]
\end{example}

Any group can be represented by a map to (possibly infinite dimensional) matrices, so a group really is just a multiplication table. A representation of a group is just a specific way of realizing the multiplication table using matrices. If we have a representation, then we can infer all properties of the group without having to examine the actual elements of the group.

Since a representation is a map from a group to the set of linear transformation, we may choose to either work with linear transformations or their matrix representations. Let $\ket{i}$ be an orthonormal basis on some space and $D(x)$ a linear operator. Then
\[ [D(g)]_{ij} = \mel{i}{D(g)}{j} \thus D(g) \ket{i} = \sum_{j} \bra{j} \mel{j}{D(g)}{i} = \sum_j \bra{j} [D(g)]_{ji}\]
This relationship allows us to freely translate between forms.

Two representations are equivalent if they are related by a similarity transform, that is there exists a matrix $S$ such that for all $x \in G$
\[ D_2(x) = S D_1(x) \inv S \]

We say a representation is reducible if it is equal to a block diagonal representation
\[ D'(x) = S D(x) \inv S = \mqty(D_1(x) & 0 \\ 0 & D_2(x) ) \]
This means that the representation space can be split into two orthogonal subspaces which are acted on by $D_1$ and $D_2$ respectively. In this case we say that $D'$ is a direct sum
\[ D' = D_1 \oplus D_2 \]
A representation is irreducible if it is not reducible, these are generally the representations we will be interested in. In physics we will assume that all representations are by unitary operators.

\subsection{Exercises}

% 1
\begin{exercise}
	Suppose $\inv x = y, z$ where $y \neq z$. Then
	\[ z = (yx) z = y (xz) = y \]
	a contradiction, so the inverse must be unique.
\end{exercise}

% 2 
\begin{exercise}
	Consider a three element group $G = \setb{e, x, y}$. The multiplication table is
	\begin{center}
	\begin{tabular}{|c|ccc|}
	\hline 
	 & e & x & y \\ 
	\hline 
	e & e & x & y \\ 
	x & x & y & e \\ 
	y & y & e & x \\ 
	\hline 
	\end{tabular} 
	\end{center}
	This is unique because we cannot have $x \cdot x = e$ since otherwise
	\[ y = (xx) y = x (xy) \thus xy = y \thus x = e \]
	a contradiction since the identity is unique, a similar argument shows why $y \cdot y \neq e$. This locks in all possible choices for the table because we must have $x \cdot x = y$. We cannot have repeat entries in a row because that would lead to a contradiction like $x = e$ or $y = e$. 
\end{exercise}

% 3
\begin{exercise}
	The vector $(1,1,1)$ is an eigenvector of every matrix and so its eigenspace will be kept invariant. We can use that vector to construct orthogonal subspaces which are acted on differently by $D$, thus the standard permutation representation is reducible.
\end{exercise}